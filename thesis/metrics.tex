\chapter{Metaevaluation and Comparison of Automatic Metrics}
\label{metrics}

\XXX{Pretavit abstrakt do nejakeho uvodu}
This paper presents the results of the WMT14 Metrics Shared Task. We asked
participants of this task to score the outputs of the MT systems involved in
WMT14 Shared Translation Task. We collected scores of 23 metrics from 12
research groups. In addition to that we computed scores of 6 standard metrics
(BLEU, NIST, WER, PER, TER and CDER) as baselines. The collected scores were
evaluated in terms of system level correlation (how well each metric's scores
correlate with WMT14 official manual ranking of systems) and in terms of
segment level correlation (how often a metric agrees with humans in comparing
two translations of a particular sentence).

Automatic machine translation metrics play a very important role in the
development of MT systems and their evaluation. There are many different
metrics of diverse nature and one would like to assess their quality. For this
reason, the Metrics Shared Task is held annually at the Workshop of Statistical
Machine Translation\footnote{\url{http://www.statmt.org/wmt13}}, starting with
\perscite{koehn-monz:2006:WMT} and following up to
\perscite{wmt14-overview-paper}.

In this task, we asked metrics developers to score the outputs of WMT14 Shared
Translation Task \parcite{wmt14-overview-paper}. We have collected the computed
metrics' scores and use them to evaluate quality of the metrics. 

The systems' outputs, human judgements and evaluated metrics are described in
Section \ref{section:data}. The quality of the metrics in terms of system level
correlation is reported in Section \ref{system-level}. Segment level
correlation with a detailed discussion and a slight change in the calculation
compared to the previous year is reported in Section \ref{segment-level}.

\section{Data}
\label{section:data}

We used the translations of MT systems involved in WMT14 Shared Translation
Task together with reference translations as the test set for the Metrics Task.
This dataset consists of 110 systems' outputs and 10 reference translations in
10 translation directions (English from and into Czech, French, German, Hindi
and Russian). For most of
the translation directions each system's output and the reference translation
contain 3003 sentences. For more details please see the WMT14 overview
paper \parcite{wmt14-overview-paper}. 

\subsection{Manual MT Quality Judgements}

During the WMT14 Translation Task, a large scale manual annotation was conducted
to compare the systems. We used these collected human judgements for the evalution
of
the automatic metrics. 

The participants in the manual annotation were asked to evaluate system outputs
by ranking translated sentences relative to each other. For each source segment
that was included in the procedure, the annotator was shown the outputs of five
systems to which he or she was supposed to assign ranks. Ties were allowed.

These collected rank labels for each five-tuple of systems were then interpreted
as 10 pairwise comparisons of systems and  used to assign each system a score
that
reflects how high that system was usually ranked by the annotators. Please see
the WMT14 overview paper for details on how this score is computed. You
can also find inter- and intra-annotator agreement estimates there.


\subsection{Participants of the Metrics Shared Task}

\begin{table*}[t]
  \small
  \begin{center}
    \begin{tabular}{rl}
      \textbf{Metric} & \textbf{Participant} \\
      \hline
      \metric{APAC} & Hokkai-Gakuen University \parcite{wmt14-metric-apac} \\
      \metric{BEER} & ILLC -- University of Amsterdam \parcite{wmt14-metric-beer} \\
      \metric{RED-*} & Dublin City University \parcite{wmt14-metric-red} \\
      \metric{DiscoTK-*} & Qatar Computing Research Institute \parcite{wmt14-metric-discotk} \\
      \metric{ELEXR} & University of Tehran \parcite{wmt14-metric-elexr} \\
      \metric{LAYERED} & Indian Institute of Technology, Bombay \parcite{wmt14-metric-layered} \\
      \metric{Meteor} & Carnegie Mellon University \parcite{wmt14-metric-meteor} \\
      \metric{AMBER, BLEU-NRC} & National Research Council of Canada \parcite{wmt14-metric-amber} \\
      \metric{Parmesan} & Charles University in Prague \parcite{wmt14-metric-parmesan} \\
      \metric{tBLEU} & Charles University in Prague \parcite{wmt14-metric-tbleu} \\
      \metric{UPC-IPA, UPC-STOUT} & Technical University of Catalunya \parcite{wmt14-metric-upc} \\
      \metric{VERTa-W, VERTa-EQ} & University of Barcelona \parcite{wmt14-metric-verta} \\
    \end{tabular}
  \end{center}
  \caption{Participants of WMT14 Metrics Shared Task}
  \label{participants}
\end{table*}

Table \ref{participants} lists the participants of WMT14 Shared Metrics Task,
along with their metrics. We have collected 23 metrics from a total of
12 research groups.

In addition to that we have computed the following two groups of standard
metrics as baselines: 

\begin{itemize}

\item \textbf{Mteval.} The metrics \metric{BLEU} \parcite{Papineni02bleu:a} and
    \metric{NIST} \parcite{Doddington:2002:NIST} were computed using the script
    \texttt{mteval-v13a.pl}\footnote{\url{http://www.itl.nist.gov/iad/mig//tools/}}
    which is used in the OpenMT Evaluation Campaign and includes its own
    tokenization.  We run \texttt{mteval} with the flag
    \texttt{-{}-international-tokenization} since it performs slightly better
    \parcite{machacek-bojar:2013:WMT}.

\item \textbf{Moses Scorer.} The metrics \metric{TER} \parcite{Snover06astudy},
    \metric{WER}, \metric{PER} and \metric{CDER} \parcite{Leusch06cder:efficient}
    were computed using the Moses scorer which is used in Moses model
    optimization. To tokenize the sentences we used the standard tokenizer
    script as available in Moses toolkit.


\end{itemize}

We have normalized all metrics' scores such that better translations get higher scores. 


\section{System-Level Metric Analysis}
\label{system-level}


While the Spearman's $\rho$ correlation coefficient was used as the main
measure of system-level metrics' quality in the past, we have decided to use
Pearson correlation coefficient as the main measure this year. At the end of
this section we give reasons for this change. 

We use the following formula to compute the Pearson's $r$ for each metric and
translation direction:

\begin{equation}
    r = \frac{\sum ^n _{i=1}(H_i - \bar{H})(M_i - \bar{M})}{\sqrt{\sum ^n _{i=1}(H_i - \bar{H})^2} \sqrt{\sum ^n _{i=1}(M_i - \bar{M})^2}} 
\end{equation}

\noindent where $H$ is the vector of human scores of all systems translating in
the given
direction, $M$ is the vector of the corresponding scores as predicted by the
given metric. $\bar{H}$ and $\bar{M}$ are their means respectively.


Since we have normalized all metrics such that better translations get higher
score, we consider metrics with values of Pearson's $r$ closer to 1 as better. 

You can find the system-level correlations for translations into English in
Table \ref{system-level-corrs-toEn} and for translations out of English in
Table \ref{system-level-corrs-outEn}. Each row in the tables contains
correlations of a metric in each of the examined translation directions. The
metrics are sorted by average Pearson correlation coefficient across
translation directions. The best results in each direction are in bold.

The reported empirical confidence intervals of system level correlations were
obtained through
bootstrap resampling of 1000
samples (confidence level of 95\,\%).

As in previous years, a lot of metrics outperformed \metric{BLEU} in system
level correlation. In into-English directions, metric
\metric{DiscoTK-party-tuned} has the highest correlation in two language
directions and it is also the best correlated metric on average according to both
Pearson and Spearman's coefficients. The second best correlated metric on
average (according to Pearson) is \metric{LAYERED} which is also the single
best metric in Hindi-to-English direction.  Metrics \metric{REDSys} and
\metric{REDSysSent} are quite unstable, they win in French-to-English and
Czech-to-English directions respectively but they perform very poorly in other
directions. 

Except \metric{Meteor}, none of the participants took part in the last year
metrics task.  We can therefore compare current and last year results only for
\metric{Meteor} and baseline metrics.  \metric{Meteor}, the last year winner,
performs generally well in some directions but it horribly suffers when
evaluating translations from non-Latin script (Russian and especially Hindi).
For the baseline metrics the results are quite similar across the years. In
both years \metric{BLEU} performs best among baseline metrics, closely followed
by \metric{CDER}. \metric{NIST} is in the middle of the list in both
years. The remaining baseline metrics \metric{TER}, \metric{WER}
and \metric{PER} perform much worse.

The results into German are markedly lower and have broader confidence
intervals than the results in other directions. This could be explained by a
very
high number (18) of participating systems of similar quality. 
Both human judgements and automatic metrics are negatively affected by these
circumstances. To preserve the reliability of overall metrics' performance
across languages, we
decided to exclude English-to-German
direction from the average Pearson and Spearman's correlation coefficients.

In other out-of-English directions, the best correlated metric on average according
to Pearson coefficient is \metric{NIST}, even though it does not win in any
single direction. \metric{CDER} is the second best according to Pearson
and the best metric according to Spearman's. Again it does not win in any
single direction. The metrics \metric{PER} and \metric{WER} are quite
unstable.  Each of them wins in two directions but performs very badly in
others.

Compared to the last year results, the order of metrics participating in both
years is quite similar: \metric{NIST} and \metric{CDER} performed very well
both years, followed by \metric{BLEU}. The metrics \metric{TER} and \metric{WER}
are again at the end of the list. An interesting change is that
\metric{PER} perform much better this year.

\subsection{Reasons for Pearson correlation coefficient}

In the translation task, there are often similar systems with human scores very
close to each other. It can therefore easily happen that even a good metric
compares two similar systems differently from humans. We believe that the
penalty incurred by the metric for such a swap should somehow reflect that the
systems were hard to separate.

Since the Spearman's $\rho$ converts both human and metric scores to ranks and
therefore disregards the absolute differences in the scores, it does exactly what
we feel is not fair. The Pearson correlation coefficient does not suffer from this 
problem. We are aware of the fact that Pearson correlation coefficient also
reflects whether the relation between manual and automatic scores is linear (as
opposed to e.g. quadratic). We don't think this would be negatively affecting
any of the metrics since overall, the systems are of a comparable quality and
the metrics are likely to behave linearly in this small
range of scores.

\XXX{pridat obrazek, ten co jsem dal do slidu}

Moreover, the general agreement to adopt Pearson instead of Spearman's correlation
coefficient was
already apparent during the WMT12 workshop. This change just did not get
through for WMT13.

\begin{sidewaystable*}
  %\small
  \begin{center}
    \begin{tabular}{r|cccccc|c}
        \textbf{Correlation coefficient} & \multicolumn{6}{|c|}{\textbf{Pearson Correlation Coefficient}} & \textbf{Spearman's} \\
        \textbf{Direction}           & \textbf{fr-en}   & \textbf{de-en}   & \textbf{hi-en}   & \textbf{cs-en}   & \textbf{ru-en}   & \textbf{Average}   & \textbf{Average}   \\
        \textbf{Considered Systems} & 8 & 13 & 9 & 5 & 13 & \\
        \hline
        \metric{DiscoTK-party-tuned} & $.977 \pm .009$        & \best{.943 $\pm$ .020} & $.956 \pm .007$        & $.975 \pm .031$        & \best{.870 $\pm$ .022} & \best{.944 $\pm$ .018} & \best{.912 $\pm$ .043} \\
        \metric{LAYERED}             & $.973 \pm .009$        & $.893 \pm .026$        & \best{.976 $\pm$ .006} & $.941 \pm .045$        & $.854 \pm .023$        & $.927 \pm .022$        & $.894 \pm .047$        \\
        \metric{DiscoTK-party}       & $.970 \pm .010$        & $.921 \pm .024$        & $.862 \pm .015$        & $.983 \pm .025$        & $.856 \pm .023$        & $.918 \pm .019$        & $.856 \pm .046$        \\
        \metric{UPC-STOUT}           & $.968 \pm .010$        & $.915 \pm .025$        & $.898 \pm .013$        & $.948 \pm .040$        & $.837 \pm .024$        & $.913 \pm .022$        & \oosmark{$.901 \pm .045$}        \\
        \metric{VERTa-W}             & $.959 \pm .011$        & $.867 \pm .029$        & $.920 \pm .011$        & $.934 \pm .050$        & $.848 \pm .024$        & $.906 \pm .025$        & $.868 \pm .045$        \\
        \metric{VERTa-EQ}            & $.959 \pm .011$        & $.854 \pm .031$        & $.927 \pm .010$        & $.938 \pm .048$        & $.842 \pm .024$        & $.904 \pm .025$        & $.857 \pm .046$        \\
        \metric{tBLEU}               & $.952 \pm .012$        & $.832 \pm .034$        & $.954 \pm .007$        & $.957 \pm .040$        & $.803 \pm .027$        & $.900 \pm .024$        & $.841 \pm .056$        \\
        \metric{BLEU\_NRC}           & $.953 \pm .012$        & $.823 \pm .035$        & $.959 \pm .007$        & $.946 \pm .044$        & $.787 \pm .028$        & $.894 \pm .025$        & \oosmark{$.855 \pm .056$}        \\
        \metric{BLEU}                & $.952 \pm .012$        & $.832 \pm .034$        & $.956 \pm .007$        & $.909 \pm .054$        & $.789 \pm .027$        & $.888 \pm .027$        & $.833 \pm .058$        \\
        \metric{UPC-IPA}             & $.966 \pm .010$        & $.895 \pm .027$        & $.914 \pm .010$        & $.824 \pm .073$        & $.812 \pm .026$        & $.882 \pm .029$        & \oosmark{$.858 \pm .044$}        \\
        \metric{CDER}                & $.954 \pm .012$        & $.823 \pm .034$        & $.826 \pm .016$        & $.965 \pm .035$        & $.802 \pm .027$        & $.874 \pm .025$        & $.807 \pm .050$        \\
        \metric{APAC}                & $.963 \pm .010$        & $.817 \pm .034$        & $.790 \pm .016$        & $.982 \pm .026$        & $.816 \pm .026$        & $.874 \pm .022$        & $.807 \pm .049$        \\
        \metric{REDSys}              & \best{.981 $\pm$ .008} & $.898 \pm .026$        & $.676 \pm .022$        & $.989 \pm .021$        & $.814 \pm .026$        & $.872 \pm .021$        & $.786 \pm .047$        \\
        \metric{REDSysSent}          & $.980 \pm .008$        & $.910 \pm .024$        & $.644 \pm .023$        & \best{.993 $\pm$ .018} & $.807 \pm .027$        & $.867 \pm .020$        & $.771 \pm .043$        \\
        \metric{NIST}                & $.955 \pm .011$        & $.811 \pm .035$        & $.784 \pm .016$        & $.983 \pm .025$        & $.800 \pm .027$        & $.867 \pm .023$        & \oosmark{$.824 \pm .055$}        \\
        \metric{DiscoTK-light}       & $.965 \pm .011$        & $.935 \pm .022$        & $.557 \pm .025$        & $.954 \pm .038$        & $.791 \pm .027$        & $.840 \pm .024$        & $.774 \pm .046$        \\
        \metric{Meteor}              & $.975 \pm .009$        & $.927 \pm .022$        & $.457 \pm .027$        & $.980 \pm .029$        & $.805 \pm .026$        & $.829 \pm .023$        & \oosmark{$.788 \pm .046$}        \\
        \metric{TER}                 & $.952 \pm .012$        & $.775 \pm .038$        & $.618 \pm .021$        & $.976 \pm .031$        & $.809 \pm .027$        & $.826 \pm .026$        & $.746 \pm .057$        \\
        \metric{WER}                 & $.952 \pm .012$        & $.762 \pm .038$        & $.610 \pm .021$        & $.974 \pm .033$        & $.809 \pm .027$        & $.821 \pm .026$        & $.736 \pm .058$        \\
        \metric{AMBER}               & $.948 \pm .012$        & $.910 \pm .026$        & $.506 \pm .026$        & $.744 \pm .095$        & $.797 \pm .027$        & $.781 \pm .037$        & $.728 \pm .051$        \\
        \metric{PER}                 & $.946 \pm .013$        & $.867 \pm .031$        & $.411 \pm .025$        & $.883 \pm .063$        & $.799 \pm .028$        & $.781 \pm .032$        & $.698 \pm .047$        \\
        \metric{ELEXR}               & $.971 \pm .009$        & $.857 \pm .031$        & $.535 \pm .026$        & $.945 \pm .044$        & $-.404 \pm .045$       & $.581 \pm .031$        & $.652 \pm .046$        \\
        \hline
    \end{tabular}
  \end{center}
  
  \caption[System-level correlations when translating into English]{
      System-level correlations of automatic evaluation metrics and the
      official WMT human scores when translating into English.  The symbol
      ``$\wr$'' indicates where the Spearman's $\rho$ average is out of
      sequence compared to the main Pearson average.}

  \label{system-level-corrs-toEn}

\end{sidewaystable*}

\begin{sidewaystable*}[t]
  %\small

  \begin{center}
    \begin{tabular}{r|ccccc|c|c}
        \textbf{Correlation coefficient}        & \multicolumn{6}{|c|}{\textbf{Pearson Correlation Coefficient}} & \textbf{Spearman's} \\
        \textbf{Direction} & \textbf{en-fr} & \textbf{en-hi} & \textbf{en-cs} & \textbf{en-ru} & \textbf{Average} & \textbf{en-de} & \textbf{Average} \\
        \textbf{Considered Systems} & 13 & 12 & 10 & 9 & & 18 & (excl. en-de)\\
        \hline
        \metric{NIST}       & $.941 \pm .022$        & $.981 \pm .006$        & $.985 \pm .006$        & $.927 \pm .012$        & \best{.959 $\pm$ .012} & $.200 \pm .046$        & \best{.850 $\pm$ .030} \\
        \metric{CDER}       & $.949 \pm .020$        & $.949 \pm .010$        & $.982 \pm .006$        & $.938 \pm .011$        & $.955 \pm .012$        & $.278 \pm .045$        & $.840 \pm .036$        \\
        \metric{AMBER}      & $.928 \pm .023$        & \best{.990 $\pm$ .004} & $.972 \pm .008$        & $.926 \pm .012$        & $.954 \pm .012$        & $.241 \pm .045$        & $.817 \pm .041$        \\
        \metric{Meteor}     & $.941 \pm .021$        & $.975 \pm .007$        & $.976 \pm .007$        & $.923 \pm .013$        & $.954 \pm .012$        & $.263 \pm .045$        & $.806 \pm .039$        \\
        \metric{BLEU}       & $.937 \pm .022$        & $.973 \pm .007$        & $.976 \pm .007$        & $.915 \pm .013$        & $.950 \pm .012$        & $.216 \pm .046$        & \oosmark{$.809 \pm .036$}        \\
        \metric{PER}        & $.936 \pm .023$        & $.931 \pm .011$        & \best{.988 $\pm$ .005} & \best{.941 $\pm$ .011} & $.949 \pm .013$        & $.190 \pm .047$        & \oosmark{$.823 \pm .037$}        \\
        \metric{APAC}       & $.950 \pm .020$        & $.940 \pm .011$        & $.973 \pm .008$        & $.929 \pm .012$        & $.948 \pm .013$        & $.346 \pm .044$        & $.799 \pm .041$        \\
        \metric{tBLEU}      & $.932 \pm .023$        & $.968 \pm .008$        & $.973 \pm .008$        & $.912 \pm .013$        & $.946 \pm .013$        & $.239 \pm .046$        & \oosmark{$.805 \pm .039$}        \\
        \metric{BLEU\_NRC}  & $.933 \pm .022$        & $.971 \pm .007$        & $.974 \pm .008$        & $.901 \pm .014$        & $.945 \pm .013$        & $.205 \pm .046$        & \oosmark{$.809 \pm .039$}        \\
        \metric{ELEXR}      & $.885 \pm .029$        & $.962 \pm .009$        & $.979 \pm .007$        & $.938 \pm .011$        & $.941 \pm .014$        & $.260 \pm .044$        & $.768 \pm .036$        \\
        \metric{TER}        & $.954 \pm .019$        & $.829 \pm .017$        & $.978 \pm .007$        & $.931 \pm .012$        & $.923 \pm .014$        & $.324 \pm .045$        & $.745 \pm .035$        \\
        \metric{WER}        & \best{.960 $\pm$ .018} & $.516 \pm .026$        & $.976 \pm .007$        & $.932 \pm .011$        & $.846 \pm .016$        & \best{.357 $\pm$ .045} & $.696 \pm .037$        \\
        \hline
        \metric{Parmesan}   & n/a                      & n/a                      & $.962 \pm .009$        & n/a                      & $.962 \pm .009$        & n/a                      & $.915 \pm .048$        \\
        \metric{UPC-IPA}    & $.940 \pm .021$        & n/a                      & $.969 \pm .008$        & $.921 \pm .013$        & $.943 \pm .014$        & $.285 \pm .045$        & $.785 \pm .050$        \\
        \metric{REDSysSent} & $.941 \pm .021$        & n/a                      & n/a                      & n/a                      & $.941 \pm .021$        & $.208 \pm .045$        & \oosmark{$.962 \pm .038$}        \\
        \metric{REDSys}     & $.940 \pm .021$        & n/a                      & n/a                      & n/a                      & $.940 \pm .021$        & $.208 \pm .045$        & $.962 \pm .038$        \\
        \metric{UPC-STOUT}  & $.940 \pm .021$        & n/a                      & $.938 \pm .011$        & $.919 \pm .013$        & $.933 \pm .015$        & $.301 \pm .044$        & $.713 \pm .040$        \\
        \hline
    \end{tabular}
  \end{center}

  \caption[System-level correlations when translating out of
  English]{System-level correlations of automatic evaluation metrics and the
      official WMT human scores when translating out of English.  The symbol
      ``$\wr$'' indicates where the Spearman's $\rho$ average is out of
      sequence compared to the main Pearson average.}

  \label{system-level-corrs-outEn}
\end{sidewaystable*}
\afterpage{\clearpage}

\section{Segment-Level Metric Analysis}
\label{segment-level}

We measure the quality of metrics' segment-level scores using Kendall's $\tau$
rank correlation coefficient. In this type of evaluation, a metric is expected
to predict the result of the manual pairwise comparison of two systems. Note
that the golden truth is obtained from a compact annotation of five systems at
once, while an experiment with text-to-speech evaluation techniques by
\perscite{vazquez-alvarez-huckvale:2002} suggests that a genuine pairwise
comparison is likely to lead to more stable results.

In the past, slightly different variations of
Kendall's $\tau$ computation were used in the Metrics Tasks. Also some of the
participants have noticed a problem with ties in the WMT13 method. Therefore, we
discuss several possible variants in detail in this paper.

\subsection{Notation for Kendall's $\tau{}$ computation}

The basic formula for Kendall's $\tau$ is:

\begin{equation}
    \tau = \frac{|Concordant| - |Discordant|}{|Concordant| + |Discordant|}
\end{equation}

\XXX{pridat priklad ze slidu}

\noindent where $Concordant$ is the set of all human comparisons for which a
given metric suggests the same order and $Discordant$ is the set of all human comparisons for
which a given metric disagrees. In the original Kendall's $\tau$, comparisons
with human or metric ties are considered neither concordant nor discordant.
However in the past, Metrics Tasks (\cite{callisonburch-EtAl:2012:WMT} and
earlier), comparisons with human ties were considered as discordant.

To easily see which pairs are counted as concordant and which as discordant, we
have developed the following tabular notation. This is for example the WMT12
method:

\begin{center}
  \begin{tabular}{cc|ccc}
                                             &     & \multicolumn{3}{c}{Metric} \\  
                  \multicolumn{2}{c|}{WMT12}       & $<$ & $=$ & $>$ \\ \hline
      \multirow{3}{*}{\rotatebox{90}{Human}} & $<$ &  1  & -1  & -1  \\
                                             & $=$ &  X  &  X  &  X  \\ 
                                             & $>$ & -1  & -1  &  1  \\ 
  \end{tabular}
\end{center}

\noindent Given such a matrix $C_{h,m}$ where $h,m \in \{<,=,>\}$\footnote{Here
the relation $<$ always means ``is better than'' even for metrics
where the better system receives a higher score.} and a metric we compute the
Kendall's $\tau$ the following way:

We insert each extracted human pairwise comparison into exactly one of the nine
sets $S_{h,m}$ according to human and metric ranks. For example the set
$S_{<,>}$ contains all comparisons where the left-hand system was ranked better
than right-hand system by humans and it was ranked the other way round by the
metric in question.

To compute the numerator of Kendall's $\tau$, we take the coefficients from the matrix
$C_{h,m}$, use them to multiply the sizes of the corresponding sets $S_{h,m}$ and
then sum them up. We do not include sets for which the value of $C_{h,m}$ is X.
To compute the denominator of Kendall's $\tau$, we simply sum the sizes of all
the sets
$S_{h,m}$ except those where $C_{h,m} = \text{X}$. To define it formally:

\begin{equation}
    \tau = \frac{
        \sum\limits_{\substack{
            h,m \in \{<,=,>\} \\
            C_{h,m} \ne \text{X}
        }}
        C_{h,m} |S_{h,m}|
    }{
        \sum\limits_{\substack{
            h,m \in \{<,=,>\} \\
            C_{h,m} \ne \text{X}
        }}
        |S_{h,m}|
    }
\end{equation}

\XXX{Pridat priklad vypoctu ze slidu}

\subsection{Discussion on Kendall's $\tau{}$ computation}

In 2013, we thought that metric ties should not be penalized and we decided to
excluded them like the human ties. We will denote this method as WMT13:

\begin{center}
  \begin{tabular}{cc|ccc}
                                             &     & \multicolumn{3}{c}{Metric} \\  
                  \multicolumn{2}{c|}{WMT13}       & $<$ & $=$ & $>$ \\ \hline
      \multirow{3}{*}{\rotatebox{90}{Human}} & $<$ &  1  &  X  & -1  \\
                                             & $=$ &  X  &  X  &  X  \\ 
                                             & $>$ & -1  &  X  &  1  \\ 
  \end{tabular}
\end{center}

\noindent It turned out, however, that it was not a good idea: metrics could
game the scoring by avoiding hard cases and assigning lots of ties. A natural
solution is to count the metrics ties also in denominator to avoid the problem.
We will denote this variant as WMT14:

\begin{center}
  \begin{tabular}{cc|ccc}
                                             &     & \multicolumn{3}{c}{Metric} \\  
                  \multicolumn{2}{c|}{WMT14}       & $<$ & $=$ & $>$ \\ \hline
      \multirow{3}{*}{\rotatebox{90}{Human}} & $<$ &  1  &  0  & -1  \\
                                             & $=$ &  X  &  X  &  X  \\ 
                                             & $>$ & -1  &  0  &  1  \\ 
  \end{tabular}
\end{center}

\noindent The WMT14 variant does not allow for gaming the scoring like the WMT13
variant does. Compared to WMT12 method, WMT14 does not penalize ties.

%has the following advantage: In WMT12
%a metric could intentionally randomly pertrubate the scores to avoid the tied
%scores. In half of the originally tied cases it would get into   moved tied comparisons from
%discordant pairs to concordant. 

We were also considering to get human ties involved. The most natural variant would be the
following variant denoted as HTIES:

\begin{center}
  \begin{tabular}{cc|ccc}
                                             &     & \multicolumn{3}{c}{Metric} \\  
    \multicolumn{2}{c|}{HTIES}                  & $<$ & $=$ & $>$ \\ \hline
      \multirow{3}{*}{\rotatebox{90}{Human}} & $<$ &  1  &  0  & -1  \\
                                             & $=$ &  0  &  1  &  0  \\ 
                                             & $>$ & -1  &  0  &  1  \\ 
  \end{tabular}
\end{center}

\noindent Unfortunately this method allows for gaming the scoring as well. The
least risky choice for metrics in hard cases would be to assign a tie because
it cannot worsen the Kendall's $\tau$ and there is quite a high chance that the
human rank is also a tie.  Metrics could be therefore tuned to predict ties
often but such metrics are not very useful. For example, the simplistic metric
which assigns the same score to all candidates (and therefore all pairs would
be tied by the metric) would get the score equal to the proportion of ties in
all human comparisons.  It would become one of the best performing metrics in
WMT13 even though it is not informative at all.

We have decided to use WMT14 variant as the main evaluation measure this year,
however, we are also reporting average scores computed by other variants.


\subsection{Kendall's $\tau$ results}

\begin{sidewaystable*}[t]
  \begin{center}
    \tiny
    \begin{tabular}{r|cccccc|ccc}
        \textbf{Direction}           & \textbf{fr-en}           & \textbf{de-en}           & \textbf{hi-en}           & \textbf{cs-en}           & \textbf{ru-en}           & \textbf{Avg}             & \multicolumn{3}{|c}{\textbf{Averages of other variants of Kendall's $\tau$}} \\
        \textbf{Extracted-pairs}     & 26090                    & 25260                    & 20900                    & 21130                    & 34460                    &                          & \textbf{WMT12}           & \textbf{WMT13}           & \textbf{HTIES}           \\
        \hline
        \metric{DiscoTK-party-tuned} & \best{.433 $\pm$ .012} & \best{.380 $\pm$ .013} & $.434 \pm .013$        & \best{.328 $\pm$ .015} & \best{.355 $\pm$ .011} & \best{.386 $\pm$ .013} & \best{.386 $\pm$ .013} & $.386 \pm .013$        & $.306 \pm .010$        \\
        \metric{BEER}                & $.417 \pm .013$        & $.337 \pm .014$        & \best{.438 $\pm$ .013} & $.284 \pm .016$        & $.333 \pm .011$        & $.362 \pm .013$        & $.358 \pm .013$        & $.363 \pm .013$        & \oosmark{\best{.318 $\pm$ .011}} \\
        \metric{REDcombSent}         & $.406 \pm .012$        & $.338 \pm .014$        & $.417 \pm .013$        & $.284 \pm .015$        & $.336 \pm .011$        & $.356 \pm .013$        & $.346 \pm .013$        & $.360 \pm .013$        & $.317 \pm .011$        \\
        \metric{REDcombSysSent}      & $.408 \pm .012$        & $.338 \pm .014$        & $.416 \pm .013$        & $.282 \pm .014$        & $.336 \pm .011$        & $.356 \pm .013$        & $.346 \pm .013$        & $.359 \pm .013$        & $.316 \pm .010$        \\
        \metric{Meteor}              & $.406 \pm .012$        & $.334 \pm .014$        & $.420 \pm .013$        & $.282 \pm .015$        & $.329 \pm .010$        & $.354 \pm .013$        & $.341 \pm .013$        & $.359 \pm .013$        & \oosmark{$.317 \pm .010$}        \\
        \metric{REDSysSent}          & $.404 \pm .012$        & $.338 \pm .014$        & $.386 \pm .014$        & $.283 \pm .015$        & $.321 \pm .010$        & $.346 \pm .013$        & $.335 \pm .013$        & $.350 \pm .013$        & $.309 \pm .010$        \\
        \metric{REDSent}             & $.403 \pm .012$        & $.336 \pm .014$        & $.383 \pm .014$        & $.283 \pm .015$        & $.323 \pm .011$        & $.345 \pm .013$        & $.334 \pm .013$        & $.349 \pm .013$        & $.308 \pm .010$        \\
        \metric{UPC-IPA}             & $.412 \pm .012$        & $.340 \pm .014$        & $.368 \pm .014$        & $.274 \pm .015$        & $.316 \pm .011$        & $.342 \pm .013$        & \oosmark{$.340 \pm .014$}        & $.343 \pm .014$        & $.300 \pm .011$        \\
        \metric{UPC-STOUT}           & $.403 \pm .012$        & $.345 \pm .014$        & $.352 \pm .014$        & $.275 \pm .015$        & $.317 \pm .011$        & $.338 \pm .013$        & $.336 \pm .013$        & $.339 \pm .013$        & $.294 \pm .011$        \\
        \metric{VERTa-W}             & $.399 \pm .013$        & $.321 \pm .015$        & $.386 \pm .014$        & $.263 \pm .015$        & $.315 \pm .011$        & $.337 \pm .014$        & $.320 \pm .014$        & \oosmark{$.342 \pm .014$}        & \oosmark{$.304 \pm .011$}        \\
        \metric{VERTa-EQ}            & $.407 \pm .013$        & $.315 \pm .014$        & $.384 \pm .013$        & $.263 \pm .015$        & $.312 \pm .011$        & $.336 \pm .013$        & \oosmark{$.323 \pm .013$}        & $.341 \pm .013$        & $.302 \pm .011$        \\
        \metric{DiscoTK-party}       & $.395 \pm .013$        & $.334 \pm .014$        & $.362 \pm .013$        & $.264 \pm .016$        & $.305 \pm .011$        & $.332 \pm .013$        & \oosmark{$.332 \pm .013$}        & $.332 \pm .013$        & $.263 \pm .011$        \\
        \metric{AMBER}               & $.367 \pm .013$        & $.313 \pm .014$        & $.362 \pm .013$        & $.246 \pm .016$        & $.294 \pm .011$        & $.316 \pm .013$        & $.302 \pm .013$        & $.321 \pm .014$        & \oosmark{$.286 \pm .011$}        \\
        \metric{BLEU\_NRC}           & $.382 \pm .013$        & $.272 \pm .014$        & $.322 \pm .014$        & $.226 \pm .016$        & $.269 \pm .011$        & $.294 \pm .013$        & $.267 \pm .014$        & $.303 \pm .014$        & $.271 \pm .011$        \\
        \metric{sentBLEU}            & $.378 \pm .013$        & $.271 \pm .014$        & $.300 \pm .013$        & $.213 \pm .016$        & $.263 \pm .011$        & $.285 \pm .013$        & $.258 \pm .014$        & $.293 \pm .014$        & $.264 \pm .011$        \\
        \metric{APAC}                & $.364 \pm .012$        & $.271 \pm .014$        & $.288 \pm .014$        & $.198 \pm .016$        & $.276 \pm .011$        & $.279 \pm .013$        & $.243 \pm .014$        & $.290 \pm .014$        & $.261 \pm .011$        \\
        \metric{DiscoTK-light}       & $.311 \pm .014$        & $.224 \pm .015$        & $.238 \pm .013$        & $.187 \pm .016$        & $.209 \pm .011$        & $.234 \pm .014$        & $.234 \pm .014$        & $.234 \pm .014$        & $.184 \pm .011$        \\
        \metric{DiscoTK-light-kool}  & $.005 \pm .001$        & $.001 \pm .000$        & $.000 \pm .000$        & $.002 \pm .001$        & $.001 \pm .000$        & $.002 \pm .001$        & $-.996 \pm .001$       & \oosmark{\best{.676 $\pm$ .256}} & \oosmark{$.211 \pm .005$}        \\
        \hline
    \end{tabular}
  \end{center}

  \caption[Segment-level correlations when translating into
  English]{Segment-level Kendall's $\tau$ correlations of automatic evaluation
      metrics and the official WMT human judgements when translating into
      English.  The last three columns contain average Kendall's $\tau$
  computed by other variants.  The symbol ``$\wr$'' indicates where the
  averages of other variants are out of sequence compared to the WMT14
  variant.}

  \label{segment-level-correlations-toEn}
%\end{sidewaystable*}


%\begin{sidewaystable*}[t]
  \begin{center}
    \tiny
    \begin{tabular}{r|cccccc|ccc}
        \textbf{Direction}      & \textbf{en-fr}           & \textbf{en-de}           & \textbf{en-hi}           & \textbf{en-cs}           & \textbf{en-ru}           & \textbf{Avg}             & \multicolumn{3}{|c}{\textbf{Averages of other variants of Kendall's $\tau$}} \\
        \textbf{Extracted-pairs}& 33350                    & 54660                    & 28120                    & 55900                    & 28960                    &                          & \textbf{WMT12}           & \textbf{WMT13}           & \textbf{HTIES}           \\
        \hline
        \metric{BEER}           & $.292 \pm .012$        & \best{.268 $\pm$ .009} & $.250 \pm .013$        & \best{.344 $\pm$ .009} & \best{.440 $\pm$ .013} & \best{.319 $\pm$ .011} & \best{.314 $\pm$ .011} & \best{.320 $\pm$ .011} & $.272 \pm .009$        \\
        \metric{Meteor}         & $.280 \pm .012$        & $.238 \pm .009$        & $.264 \pm .012$        & $.318 \pm .009$        & $.427 \pm .012$        & $.306 \pm .011$        & $.283 \pm .011$        & $.313 \pm .011$        & \oosmark{\best{.273 $\pm$ .008}} \\
        \metric{AMBER}          & $.264 \pm .012$        & $.227 \pm .009$        & \best{.286 $\pm$ .012} & $.302 \pm .009$        & $.397 \pm .013$        & $.295 \pm .011$        & $.269 \pm .011$        & $.303 \pm .011$        & $.266 \pm .009$        \\
        \metric{BLEU\_NRC}      & $.261 \pm .012$        & $.202 \pm .009$        & $.234 \pm .013$        & $.297 \pm .009$        & $.391 \pm .012$        & $.277 \pm .011$        & $.235 \pm .011$        & $.289 \pm .011$        & $.256 \pm .009$        \\
        \metric{APAC}           & $.253 \pm .012$        & $.210 \pm .008$        & $.203 \pm .012$        & $.292 \pm .009$        & $.388 \pm .013$        & $.269 \pm .011$        & $.217 \pm .011$        & $.285 \pm .011$        & $.252 \pm .008$        \\
        \metric{sentBLEU}       & $.256 \pm .012$        & $.191 \pm .009$        & $.227 \pm .012$        & $.290 \pm .009$        & $.381 \pm .013$        & $.269 \pm .011$        & \oosmark{$.232 \pm .011$}        & $.280 \pm .011$        & $.246 \pm .009$        \\
        \hline
        \metric{UPC-STOUT}      & $.279 \pm .011$        & $.234 \pm .008$        & n/a                      & $.282 \pm .009$        & $.425 \pm .013$        & $.305 \pm .011$        & $.300 \pm .010$        & $.306 \pm .011$        & $.256 \pm .008$        \\
        \metric{UPC-IPA}        & $.264 \pm .012$        & $.227 \pm .009$        & n/a                      & $.298 \pm .009$        & $.426 \pm .013$        & $.304 \pm .011$        & $.292 \pm .011$        & \oosmark{$.308 \pm .011$}        & \oosmark{$.259 \pm .008$}        \\
        \metric{REDSent}        & \best{.293 $\pm$ .012} & $.242 \pm .009$        & n/a                      & n/a                      & n/a                      & $.267 \pm .010$        & $.246 \pm .010$        & $.273 \pm .011$        & $.257 \pm .008$        \\
        \metric{REDcombSysSent} & $.291 \pm .012$        & $.244 \pm .009$        & n/a                      & n/a                      & n/a                      & $.267 \pm .010$        & \oosmark{$.249 \pm .010$}        & $.272 \pm .010$        & $.256 \pm .008$        \\
        \metric{REDcombSent}    & $.290 \pm .012$        & $.242 \pm .009$        & n/a                      & n/a                      & n/a                      & $.266 \pm .010$        & $.248 \pm .010$        & $.271 \pm .011$        & $.256 \pm .008$        \\
        \metric{REDSysSent}     & $.290 \pm .012$        & $.239 \pm .008$        & n/a                      & n/a                      & n/a                      & $.264 \pm .010$        & $.235 \pm .010$        & \oosmark{$.273 \pm .010$}        & \oosmark{$.257 \pm .008$}        \\
        \hline
    \end{tabular}
  \end{center}

  \caption[Segment-level correlations when translating out of
  English]{Segment-level Kendall's $\tau$ correlations of automatic evaluation
      metrics and the official WMT human judgements when translating out of
      English.  The last three columns contain average Kendall's $\tau$
  computed by other variants.  The symbol ``$\wr$'' indicates where the
  averages of other variants are out of sequence compared to the WMT14
  variant.}

  \label{segment-level-correlations-fromEn}
\end{sidewaystable*}

\afterpage{\clearpage}




The final Kendall's $\tau$ results are shown in Table
\ref{segment-level-correlations-toEn} for directions into English and in Table
\ref{segment-level-correlations-fromEn} for directions out of English.  Each
row in the tables contains correlations of a metric in given directions. The
metrics are sorted by average correlation across translation directions.  The
highest correlation in each column is in bold.  The tables also contain average
Kendall's $\tau$ computed by other variants including the variant WMT13 used
last year.  Metrics which did not compute scores in all directions
are at the bottom of the tables. The possible values of $\tau$ range between -1
(a metric always predicted a different order than humans did) and 1 (a metric
always predicted the same order as humans). Metrics with a higher $\tau$ are
better.

We also computed empirical confidence intervals of Kendall's $\tau$ using bootstrap
resampling. We varied the ``golden truth'' by sampling from human judgments. We
have generated 1000 new sets and report the average of the upper and lower
2.5\,\% empirical bound, which corresponds to the 95\,\% confidence interval.
%XXX Matousi, je tenhle popis srozumitelny?




In directions into English (Table~\ref{segment-level-correlations-toEn}), the
strongest correlated segment-level metric on average is
\metric{DiscoTK-party-tuned} followed by \metric{BEER}. Unlike the system level
correlation, the results are much more stable here.
\metric{DiscoTK-party-tuned} has the highest correlation in 4 of 5 language
directions. Generally, the ranking of metrics is almost the same in each
direction. 

The only two metrics which also participated in last year metrics task are
\metric{Meteor} and \metric{sentBLEU}. In both years, \metric{Meteor} performed
quite well unlike \metric{sentBLEU} which was outperformed by most of the
metrics. 

The metric \metric{DiscoTK-light-kool} is worth mentioning.  It is deliberately
designed to assign the same score for all systems for most of the segments.  It
obtained scores very close to zero (i.e. totally uninformative) in WMT14
variant.  In WMT13 thought it reached the highest score.

In directions out of English (Table~\ref{segment-level-correlations-fromEn}),
the metric with highest correlation on average across all directions is
\metric{Beer}, followed by \metric{Meteor}.

\section{Overall Comparison of Automatic Metrics}

In this paper, we summarized the results of the WMT14 Metrics Shared Task, which
assesses the quality of
various automatic machine translation metrics. As in previous years, human
judgements collected in WMT14 serve as the golden truth and we check how well
the metrics predict the judgements at the level of individual sentences as well
as at the level of the whole test set (system-level).

This year, neither the system-level nor the segment-level scores are directly
comparable to the previous years. The system-level scores are affected by the
change of the underlying interpretation of the collected judgements in the main
translation task evaluation as well as
our choice of Pearson coefficient instead of Spearman's rank correlation. The
segment-level scores are affected by the different handling of ties this year.
Despite somewhat sacrificing the year-to-year comparability, we believe all
changes are towards a fairer evaluation and thus better in the long term.

As in previous years, segment-level correlations are much lower than
system-level ones, reaching at most Kendall's $\tau$ of 0.45 for the best
performing metric in its best language pair. So there is quite some research
work to be done. We are happy to see that many new metrics emerged this year,
which also underlines the importance of the Metrics Shared Task.


