Název práce:
Měření kvality strojového překladu

Autor:
Matouš Macháček

Ústav:
Ústav formální a aplikované lingvistiky

Vedoucí diplomové práce:
RNDr. Ondřej Bojar, Ph.D., UFAL

Abstrakt: V této práci zkoumáme manuální i automatické metody pro vyhodnocování
kvality strojového překladu. Navrhujeme manální metodu evaluace, ve které
anotátoři hodnotí místo celých vět pouze krátké úseky strojového překladu, což
zjednodušuje a zefektivňuje anotaci. Provedli jsme anotační experiment a
vyhodnotili jsme systémy strojového překladu podle této metody. Získané
výsledky jsou velmi podobné těm z oficiálního vyhodnocení systémů v rámci
sotěže WMT14. Získanou databázi anotací dále používáme k evaluaci nových,
neviděných systému a k ladění parametrů statistického strojového překladače.
Evaluace nových systémů ale dává nepřesné výsledky a v práci proto analyzujeme
důvody tohoto neúspěchu. V rámci zkoumání automatických metod evaluace jsme
dvakrát po sobě organizovali soutěž strojových metrik v rámci workshopu WMT.  V
této práci uvádíme výsledky z poslední soutže, diskutujeme různé metody
metaevaluace a analyzujeme některé zúčastněné metriky. 

Klíčová slova:
strojový překlad, vyhodnocování kvality, automatické metriky, anotace
