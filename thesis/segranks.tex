\chapter{Proposed Semi-Automatic Evaluation Method}
\XXX{Zopakovat motivaci, predstaveni metody}

The method we propose consists of two part. The first part is a way how humans
will judge outputs of judged systems. The second part is how to interpret
collected judgments to compute overall scores and rank the systems. We discuss
these two parts in following two sections.

In the WMT official human evaluation humans judge whole sentences. They get
five candidate translations of a given source sentence and their task is to
rank these candidates relatively (ties are allowed). One of disadvantages of
this method is that sentences are quite long and therefore quite hard to
remember for judge to compare them. Also when comparing longer sentences there
are much more aspects in which one sentence can be better or worse than second
sentence and therefore it is more difficult for judges to choose a winner. 

To avoid these disadvantages we take inspiration from the work of \XXX{citovat
clanek ktery navrhuje anotovani segmentu}. Instead of judging whole sentences
we extract shorter segments from candidates and give them to judges to rank
them. In order to extract coresponding segments from all candidates we first
extract short segment from the source sentence and then use an automatic alignments
to project the short segment their counterparts in the candidates.


\section{Data and Segment Preparation}
\XXX{Jaka data jsem pouzil, preprocessing, jak jsem extrahoval segmenty}

\section{Segments Ranking}
\XXX{Annotacni prostredi, instrukce k anotovani, prubeh anotace, statistiky
analyza ziskane databaze, mezianotatorske shody}

\section{Experiments}

\subsection{Evaluating Annotated Systems}
\XXX{Zde uvedu vzorecek (mozna vice variant) pro vypocet skore systemu, ktere
byly anotovane. Vypocet skore, vypocet korelace s oficialnimi WMT14 vysledky.
Porovnani obou metod z hlediska mnozstvi lidske prace.}

\subsection{Evaluating New Systems}
\XXX{Zde zkusim pouzit vytvorenou databazi pro vyhodnoceni noveho systemu
(ktery vsak nekde musim ziskat)}

\subsection{Tuning Systems}
\XXX{Zde zkusim pouzit vytvorenou databazi pro MT tuning}

\section{Comparison to Other Manual Methods}
