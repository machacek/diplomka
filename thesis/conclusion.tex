\chapter{Conclusion}
\label{chapter:conclusion}

In this thesis, we proposed a new method for manual evaluation, called
\metoda{SegRanks}, in which annotators rank only a short segments (up to six
words) of a translated sentence. The ranking of short segments is easier for
annotators, since the they do not have to read and remember whole sentences at
once. The most promising benefit of this method is that short segments are
often translated identically.  We can take advantage of this in two ways:
First, annotators are shown identical segments only once so that they do not
have to rank them multiple times. (In our experiment, we reduced the number of
segments to rank almost two times \XXX{overit, upresnit}). Second, the
evaluated segments can be stored together with their ranks in a database, which
can be used later to automatically evaluate unseen sentences or to tune a
system's parameters. We also discussed disadvantages of this method. The most
severe ones are that the extracted segments do not always cover the whole
sentence and that the segments are evaluated without their sentence context.

We developed an easy-to-use and modern annotation interface and conducted a
manual evaluation experiment using the proposed method. We evaluated the
systems which participated in the English-Czech direction in WMT Translation
Task. The measured the inter- and intra-annotator kappa scores (the normalized
versions of agreement) are higher than the corresponding values in the WMT
manual evaluation.




\section{Future Work}
