\documentclass[12pt,a4paper]{report}

%% Verze pro jednostranný tisk:
% Okraje: levý 40mm, pravý 25mm, horní a dolní 25mm
% (ale pozor, LaTeX si sám přidává 1in)
\setlength\textwidth{145mm}
\setlength\textheight{247mm}
\setlength\oddsidemargin{15mm}
\setlength\evensidemargin{15mm}
\setlength\topmargin{0mm}
\setlength\headsep{0mm}
\setlength\headheight{0mm}

% \openright zařídí, aby následující text začínal na pravé straně knihy
\let\openright=\clearpage

%% Použité kódování znaků
\usepackage[utf8]{inputenc}

%% Ostatní balíčky
\usepackage{graphicx}
\usepackage{amsmath}
\usepackage{amssymb}
\usepackage{latexsym}
\usepackage{mathtools}
%\usepackage{amsfonts}
%\usepackage{algorithm2e}
\usepackage{url}
\usepackage{afterpage}
\usepackage{natbib}
\usepackage{rotating}
\usepackage{multirow}
%\usepackage{multicol}
\usepackage{tocbibind}
\usepackage{color}

%% Balíček hyperref, kterým jdou vyrábět klikací odkazy v PDF,
%% ale hlavně ho používáme k uložení metadat do PDF (včetně obsahu).
\usepackage[hidelinks]{hyperref}
\hypersetup{pdftitle=Measures of Machine Translation Quality}
\hypersetup{pdfauthor=Matouš Macháček}

% Tato makra přesvědčují mírně ošklivým trikem LaTeX, aby hlavičky kapitol
% sázel příčetněji a nevynechával nad nimi spoustu místa. Směle ignorujte.
%\makeatletter
%\def\@makechapterhead#1{
%  {\parindent \z@ \raggedright \normalfont
%   \Huge\bfseries \thechapter. #1
%   \par\nobreak
%   \vskip 20\p@
%}}
%\def\@makeschapterhead#1{
%  {\parindent \z@ \raggedright \normalfont
%   \Huge\bfseries #1
%   \par\nobreak
%   \vskip 20\p@
%}}
%\makeatother

\def\parcite#1{\citep{#1}}
\def\perscite#1{\cite{#1}}

% Deklarace stylů fontů pro různé druhy termínů
\newcommand{\metric}[1]{\textsc{#1}}
\newcommand{\system}[1]{\textsc{#1}}
\newcommand{\pojem}[1]{\texttt{#1}}
\newcommand{\script}[1]{\texttt{#1}}
\newcommand{\XXX}[1]{\textcolor{red}{XXX #1}}
%\newcommand{\XXX}[1]{}
\newcommand{\best}[1]{\textbf{#1}}
\def\oosmark#1{\llap{$\wr$\,}#1}  % out-of-sequence mark


\begin{document}

% Trochu volnější nastavení dělení slov, než je default.
\lefthyphenmin=2
\righthyphenmin=2

%%% Titulní strana práce

\pagestyle{empty}
\begin{center}

\large

Charles University in Prague

\medskip

Faculty of Mathematics and Physics

\vfill

{\bf\Large MASTER THESIS}

\vfill

\centerline{\mbox{\includegraphics[width=60mm]{img/logo}}}

\vfill
\vspace{5mm}

{\LARGE Matouš Macháček}

\vspace{15mm}

% Název práce přesně podle zadání
{\LARGE\bfseries Measures of \\ Machine Translation Quality}

\vfill

% Název katedry nebo ústavu, kde byla práce oficiálně zadána
% (dle Organizační struktury MFF UK)
Institute of Formal and Applied Linguistics

\vfill

\begin{tabular}{rl}

Supervisor of the master thesis: & RNDr. Ondřej Bojar, Ph.D. \\
\noalign{\vspace{2mm}}
Study programme: & Computer Science \\
\noalign{\vspace{2mm}}
Specialization: & Mathematical Linguistics \\
\end{tabular}

\vfill

% Zde doplňte rok
Prague 2014

\end{center}

\newpage

%%% Následuje vevázaný list -- kopie podepsaného "Zadání diplomové práce".
%%% Toto zadání NENÍ součástí elektronické verze práce, nescanovat.

%%% Na tomto místě mohou být napsána případná poděkování (vedoucímu práce,
%%% konzultantovi, tomu, kdo zapůjčil software, literaturu apod.)

\openright

\noindent
\XXX{Dedication.}

\newpage

%%% Strana s čestným prohlášením k diplomové práci

\vglue 0pt plus 1fill

\noindent
I declare that I carried out this master thesis independently, and only with the cited
sources, literature and other professional sources.

\medskip\noindent
I understand that my work relates to the rights and obligations under the Act No.
121/2000 Coll., the Copyright Act, as amended, in particular the fact that the Charles
University in Prague has the right to conclude a license agreement on the use of this
work as a school work pursuant to Section 60 paragraph 1 of the Copyright Act.

\vspace{10mm}

\hbox{\hbox to 0.5\hsize{%
In Prague, 31th of July 2014 
\hss}\hbox to 0.5\hsize{%
%signature of the author
\hss}}

\vspace{20mm}
\newpage

%%% Povinná informační strana diplomové práce

\vbox to 0.5\vsize{
\setlength\parindent{0mm}
\setlength\parskip{5mm}

Název práce:
Měření kvality strojového překladu

Autor:
Matouš Macháček

Ústav:
Ústav formální a aplikované lingvistiky

Vedoucí diplomové práce:
RNDr. Ondřej Bojar, Ph.D., UFAL

Abstrakt:
\XXX{Napsat Abstrakt v cestine}

Klíčová slova:
strojový překlad, vyhodnocování kvality, automatické metriky, anotace

\vss}\nobreak\vbox to 0.49\vsize{
\setlength\parindent{0mm}
\setlength\parskip{5mm}

Title:
Measures of Machine Translation Quality

Author:
Matouš Macháček

Department:
Institute of Formal and Applied Linguistics

Supervisor:
RNDr. Ondřej Bojar, Ph.D., UFAL

Abstract:
\XXX{Napsat Abstrakt v anglictine}

Keywords:
machine translation, quality measurement, automatic metrics, annotation

\vss}

\newpage

%%% Strana s automaticky generovaným obsahem diplomové práce. U matematických
%%% prací je přípustné, aby seznam tabulek a zkratek, existují-li, byl umístěn
%%% na začátku práce, místo na jejím konci.

\openright
\pagestyle{plain}
\setcounter{page}{1}
\tableofcontents

%%% Jednotlivé kapitoly práce jsou pro přehlednost uloženy v samostatných souborech
\chapter{Introduction}

The field of machine translation (MT) experienced a very fast development over
the past twenty years. It was primarily caused by the growing power of
computers which allowed researchers to start using statistical methods which
require a lot of computer resources to both learn statistics from data and then
to use them in translation.

In the globalized world we live in, there is a need for translating from one
language to another. The translation itself is often not easy even for people
who have to be trained for that and therefore well paid. There is therefore a
very high demand for cheap and high-quality machine translation. A lot of
researchers and companies try to satisfy this demand and constantly improve
their translations system. The first presumption for improving your system is
that you know how to measure the improvement. For this reason, we explore the
area of measuring machine translation quality in this thesis.

Unlike other applications of machine learning, the evaluation in machine
translation is not easy at all. The main reason for that is that when
translating an average sentence there is no single correct solution, in fact
there are hundreds of thousands correct translations
\parcite{bojar2013scratching}. Unlike other problems




\begin{itemize}
  \item zminit prekotny vyvoj strojoveho prekladu
  \item proc je potreba merit kvalitu
  \item zakladni pristupy mereni kvality, manualni, automaticke, dalsi zpusoby 
\end{itemize}




\begin{comment}
Machine Translation quality can be measured in two different ways: using human
evaluation or automatic metrics.  Altough human evaluation is considered more
accurate than automatic evaluation, it suffers some disadvantages.  It is slow
and expensive and therefore cannot be used in tuning parameters of statistical
models.

The aim of this thesis is to develop a new semi-automatic evaluation measure,
which would have advantages of both human evaluation and automatic evaluation.
The idea is to evaluate small sentence segments by human and create a database
of such annotation which could be used later to automatically evaluate new
unseen sentences.  This new measure should evaluate MT outputs more similarly
to how human do and still be cheap and fast after the initial database of
annotations is created once. It could be therefore used in tuning parameters of
MT systems.

The important part of the thesis is to design and develop a new annotation
environment which will be used to collect the annotations. This will be then used
to annotate wmt14 test data. The new measure will be analysed and compared to
current manual and automatic evaluation measures. 
\end{comment}

\section{Motivation and Goals}

\section{Outline}

\XXX{\parcite{wmt14-overview-paper}}
\XXX{\parcite{bojar2012cestina}}





\chapter{Related Work}
\label{chapter:related}

This chapter surveys related work on the boundary of automatic and manual
evaluation. At the end, we also report related work to the automatic metric
evaluation.

\section{Feasibility of Human Evaluation in MERT}

The work of \perscite{human-in-the-loop} was the main inspiration for our
\metoda{SegRanks} method. They develop a new metric called \metric{Rypt} to use
it primarily in the MERT method. This metric takes human judgments into
account, but requires manual labour only at the beginning to build a database
that can be reused later to evaluate unseen candidates. The core idea is to
extract segments from source parsed tree and then using an alignment produced
by a decoder project these source side segments to segments in n-best list
candidates.  The target side segments are then evaluated by humans and stored
to a database, which is used later when scoring n-best list. The authors claim
that this evaluation is done only once before the first iteration of MERT,
however they do not specify how new, unseen segments from n-best lists produced
in later MERT iterations would be evaluated.

Despite the \metric{Rypt} metric is designed to be used in the MERT method,
\perscite{human-in-the-loop} actually have not done any experiment with MERT
for a lack of resources. Only a pilot study is reported in the paper. They
tried the method only on a relatively small sample of sentences from n-best
list produced with already tuned weights. The reason why we could afford to do
the experiment with MERT with comparable resources is that we do not extract
candidate segments from the whole n-best list.

From their paper, we adopted mainly the short segment candidate extraction
process. The annotation process, scoring the candidates and conducted
experiments are, however, quite different to our work. The main difference is
that they extract the short segments for evaluation directly from an n-best
list, while we extract them from the evaluated systems' translations and hope
that they will cover also the n-best list. The difference in the annotating
short segments is that annotators in the paper of \perscite{human-in-the-loop}
do not rank candidate segments relatively to each other, but they use absolute
labels \texttt{YES}, \texttt{NO} and \texttt{NOT SURE} to judge whether a
candidate segment is an acceptable translation. The next difference is in the
scoring, while we compute \metoda{Ratio of wins (ignoring ties)}, they compute
the proportion of short segments labeled \texttt{YES}.  We decided to do these
changes in our method to have the annotation more similar to the official WMT
human evaluation.

\section{Extrapolating Score from Similar Sentences}

\perscite{niessen2000evaluation} have developed a tool for manual evaluation.
Annotators select for each evaluated sentence a rank from an absolute point
scale. Each evaluated sentence is then stored to a database with its rank. The
authors use their tool for everyday evaluating of new variants of their system
which often translate differently only a small percentage of a development test
set\footnote{This paper was actually published before the MERT method was
introduced.  When it is used, it changes most of the translations.}.
Identically translated sentences are therefore not evaluated again and are
automatically assigned a rank from the database. Only the new translations are
evaluated by humans and stored into the database with their rank.

When the database is large enough, there is an option to evaluate new
translations automatically by extrapolating ranks of candidates from the
database.  For an evaluated candidate sentence, the rank of the closest
sentence by edit distance is assigned. If there is more sentences in the
database with equal edit distance, the average rank is used. This is similar to
the matching the closest segment which we do in Section
\ref{match:editdistance}.

The authors present a few statistics related to their database, such as an
average of absolute differences between the real score and the extrapolated
score computed using the method similar to our \metoda{leave-one-out} trick.
However, they do not show how good the extrapolated scores are and if they also
do not suffer from overestimation. One of their collected database contains
42.9 candidate translations per a source sentence on average. This is much
higher than in our database (the maximum number of candidates for one source
segment is 10), so we could speculate that their space of candidates is much
more dense and therefore may not be so affected by the overestimation.

\section{Scratching the surface of possible translations}

The work by \perscite{bojar2013scratching} is quite different to the previous
two works. Their longterm goal is to improve automatic evaluation by
significantly enlarging the set of reference translations. Any metric that can
compare a candidate to multiple references can be then used for evaluation. The
idea is that if we have a very large set of references, then there will be higher
chance that either the evaluated candidate will be in the reference set, or
there will be a reference very similar to the candidate. In both of the cases,
an automatic metric will predict the quality much more accurately. 

To systematically construct the very large set of reference translations,
\perscite{bojar2013scratching} propose compact representation in which
annotators create many translations of smaller units, called bubbles, and
specify conditions under which the translated bubbles can combine together to
create the whole reference translation. All possible combinations are generated
and added to the set of reference translations. A single annotator could for a
given source sentence produce hundreds of thousands reference translations
using this method in two hours of work. 

The authors show that BLEU computed on a test set of 50 sentences with all the
produced references achieves better correlation with human judgments than BLEU
computed on a test set of 3003 sentences with single reference translation. It
would be interesting to experiment with many references when tuning a system
using MERT method.

\subsection{Metaevaluation}

Metrics Shared Task (also sometimes called Evaluation Task) is held annually
within Workshop on Statistical Machine Translation starting by
\perscite{wmt08}. Until the year 2012, the tasks' results used to be reported
in the main overview paper.  In the years 2013 and 2014, it was organized by
\perscite{machacek:2013} and \perscite{machacek:2014} and reported in dedicated
papers. 

Besides the shared task within WMT, there were also MetricsMATR evaluation
campaign in years
2008\footnote{\url{http://www.itl.nist.gov/iad/mig/tests/metricsmatr/2008/}}
and
2010\footnote{The task was joint with the WMT task this year, \url{http://www.nist.gov/itl/iad/mig/metricsmatr10.cfm}}.


\chapter{Proposed Semi-Automatic Evaluation Method}

The method we propose consists of two parts. The first part is a way how humans
judge outputs of judged systems. The second part is how to interpret collected
judgments to compute overall scores and rank the systems. We discuss these two
parts in following two sections. \XXX{V tomto odstavci (a asi i v dalsich odstavcich
neni dostatecne zdurazneno, proc je to semi-automaticka metoda, mozna opustit
oznaceni semi-automaticka a pouzit misto toho neco jako Manual Evaluation Method
with posibility to reuse collected judgementes for new systems)}

In the WMT official human evaluation humans judge whole sentences. They get
five candidate translations of a given source sentence and their task is to
rank these candidates relatively (ties are allowed). One of disadvantages of
this method is that sentences are quite long and therefore quite hard to
remember for judge to compare them. Also when comparing longer sentences there
are much more aspects in which one sentence can be better or worse than second
sentence and therefore it is more difficult for judges to choose a winner. 

\begin{algorithm}[H]
    \KwData{Tree, MaxSegmentLength}
    \KwResult{List of extracted segments}
    \If{Tree covers more than MaxSegmentLength}{
      yield all leaves \;
    }
    \caption{Segment extraction from parsed tree}
    \label{segment:extraction}
\end{algorithm}

To avoid these disadvantages we propose the following method. Instead of
judging whole sentences we extract shorter segments from candidates and give
them to judges to rank them. In order to extract meaningful segments with the
same meaning from all candidates we do the following procedure: First we parse
the source sentence and then we go recursively down the parsed tree and find
nodes which covers source segments with given maximum length (which is a
parameter of this method). This is exactly described in algorithm
\ref{segment:extraction}. Finally we project these extracted source segments to
their counterpart segments in all candidate sentences using an automatic
alignment.  You can find the whole process ilustrated in figure \XXX{nakreslit
obrazek}.  This extraction method is inspired by \XXX{citovat ten clanek a
stary WMT clanek}.

In \XXX{citovat WMT}, these extracted segments are only highlighted and shown
to judges together with the rest of the sentence. Judges are asked to rank the
segments in the context of whole sentences. \XXX{Zkontrolovat presne instrukce
z WMT}

We use different approach here which is more similar to that used in
\XXX{citovat ten clanek}. We show the extracted segments without any context
and ask judges to rank them. The only additional information provided to
annotators is the whole source sentence with the source segment highlighted.
Judges are told that they can imagine the rest of the sentence in which the
ranked segment fits best. They are instructed to penalize only those segments
for which they cannot imagine any appropriate rest of the sentence.

While we are aware that this approach has some disadvantages (which we
summarize bellow) there is one significant advantage: it is much more likely
that two systems produce the same translation of a short segment then they
would produce the same translation of a whole sentence. Because we do not show
the sentence context to annotators we can merge the equal segment candidates
into one, so the annotators have less candidate segments to rank. This also
allows us to reuse already collected human judgements later to evaluate a new
system which was not in the set of annotated systems (we present this
experiment in the section \ref{evaluating-new-systems}) or to tune parameters
of a system (we present this experiment in the section \ref{tuning-systems}).

\section{Data and Segment Preparation}

We have conducted an annotation experiment using the proposed method. We used
English to Czech part of the WMT14 \XXX{citovat} test set. We choose this data
set to be able to compare experiments' results with the official WMT14 human
evaluation. 

The testset consists of 3003 \XXX{overit} sentences. It contains both source
sentences and reference translations. Roughly a half of the sentences was
originally in Czech and was translated by human translators into English. The
second half of the sentences was translated in opposite direction. Besides the
source and reference translations, we also used candidate translations of 10
systems which participated in the WMT14 translation task. All systems are listed in 
the table \ref{translation-task-participants}


\begin{table}[h]
  \small
  \begin{center}
    \begin{tabular}{|l|l|l|}
      \hline
      \textbf{ID} & \textbf{Type} & \textbf{Team} \\
      \hline
      \system{cu-depfix} & statistical & \multirow{4}{*}{Charles University, Prague \XXX{(Tamchyna et al., 2014)}}  \\
      \system{cu-bojar} & statistical &  \\
      \system{cu-funky} & statistical &  \\
      \system{cu-tecto} & statistical &  \\
      \hline
      \system{uedin-phrase} & statistical &  \multirow{2}{*}{University of Edinburgh \XXX{(Durrani er al., 2014b)}} \\
      \system{uedin-uncnstr} &  statistical &  \\
      \hline
      \system{commercial-1} & rule-based & \multirow{2}{*}{Commercial machine translation systems} \\
      \system{commercial-2} & rule-based & \\
      \hline
      \system{online-a} & statistical & \multirow{2}{*}{Online statistical machine translation systems} \\
      \system{online-b} & statistical & \\
      \hline
    \end{tabular}
  \end{center}
  \caption{Systems participating WMT14 translation task in direction English-Czech \XXX{Zkontrolovat typy nekterych systemu}}
  \label{translation-task-participants}
\end{table}






\XXX{Jaka data jsem pouzil, preprocessing, jak jsem extrahoval segmenty}

\section{Segments Ranking}
\XXX{Annotacni prostredi, instrukce k anotovani, prubeh anotace, statistiky
analyza ziskane databaze, mezianotatorske shody}

\section{Experiments}

\subsection{Evaluating Annotated Systems}
\XXX{Zde uvedu vzorecek (mozna vice variant) pro vypocet skore systemu, ktere
byly anotovane. Vypocet skore, vypocet korelace s oficialnimi WMT14 vysledky.
Porovnani obou metod z hlediska mnozstvi lidske prace.}

\subsection{Evaluating New Systems}
\label{evaluating-new-systems}

\XXX{Zde zkusim pouzit vytvorenou databazi pro vyhodnoceni noveho systemu}

\XXX{provest experimenty podobne tem z clanku An Evaluation Tool for Machine
Translation: Fast Evaluation for MT Research}

\subsection{Tuning Systems}
\label{tuning-systems}

\XXX{Zde zkusim pouzit vytvorenou databazi pro MT tuning}

\section{Comparison to Other Manual Methods}

\chapter{Metaevaluation and Comparison of Automatic Metrics}

\XXX{Tady v te casti bych chtel co nejvice vytezit praci na metrics tasku. Muzu
sem preklopit cele sekce tak, jak jsou v clanku? Muzu to vubec udelat, kdyz na clanku jsme pracovali oba a nektere
casti si psal Ty? Nektere casti bych chtel rozsirit, napriklad popsat strucne vsechny zucastnene metriky a v posledni sekci metriky subjektivne porovnat, shrnout jejich plusy, minusy, atd.}

\section{Data}
\subsection{Manual MT Quality Judgements}
\subsection{Participants of the Metrics Shared Task}
\section{System-Level Metric Analysis}
\subsection{Reasons for Pearson correlation coefficient}
\section{Segment-Level Metric Analysis}
\subsection{Notation for Kendall's $\tau{}$ computation}
\subsection{Discussion on Kendall's $\tau{}$ computation}
\subsection{Kendall's $\tau$ results}
\section{Overall Comparison of Automatic Metrics}

\chapter{SegRanks Application Development Documentation}
\label{chapter:implementation}

\textit{SegRanks} is a Django web application which follows the standard
Django's structure and guidelines. It uses Model-View-Controller (MVC) pattern,
although in Django, views are called templates and controllers are called
views. We will use the Django's terminology here.

The model describes the representation of the data stored in the database.
Views prepare the data that gets presented to the user and also process user
requests and updates the data. Templates describe how the presented data will
look.

\section{Model}

The Django's object-relational mapping (ORM) is used to access the data in the
database. It allows to manipulate with data using an object interface. There
is no need to write SQL queries manually, all manipulations with data objects are
translated to SQL automatically on background. 

The database model is implemented in \texttt{segranks/model.py} using the
Django's model API. You can see the model illustrated in Figure \ref{model}.

Annotation projects are stored in table \textit{RankProject}. Each project
contains a number of sentences stored in table \textit{Sentence}, together with
reference translations. 

All extracted segments of a sentence are stored in table \textit{Segment}.
This table has two important fields. The first is \textit{candidates\_str},
which stores tab separated strings of all candidate translations of the
segment.  Although this is not a normalized design, it makes a lot of things
much simpler (all the candidates are ranked at once anyway).  The second field
is \textit{segment\_indexes} which stores the indices of words of the segment
in the source sentence.  This is used for highlighting the segment in the
interface.

Finally, each segment has zero or more annotations stored in
\textit{Annotation} table. When an annotation is submitted to the server, a new
row for each segment in the sentence is created in this table. The most
important field in this table is \textit{ranks} which stores ranks of the
segment candidates separated by tabs as a string. The reason for breaking the
database normalization is the same as before. There are convenient getters and
setters for these unnormalized data, so this is only an implementation detail,
hidden from the rest of the application. The table also stores some metadata
(who created the annotation, how long did it take, etc.).

Because there were some changes of the database model during the development,
we use \textit{South} to track the database changes and easily migrate the
data.

\begin{figure}
    \begin{center}
        \includegraphics[width=12cm]{img/model.pdf}
    \end{center}

    \caption{Database model}
    \label{model}
\end{figure}

\section{Views}

\section{Templates}






\chapter{Conclusion}
\label{chapter:conclusion}

In this thesis, we proposed a new method for manual evaluation, called
\metoda{SegRanks}, in which annotators rank short segments (up to six words) of
a translated sentence relatively to each other. The ranking of short segments
is easier for annotators, since the they do not have to read and remember whole
sentences at once. The most promising benefit of this method is that short
segments are often translated identically.  We can take advantage of this in
two ways: First, annotators are shown identical segments only once so that they
do not have to rank them multiple times. (In our experiment, we reduced the
number of segments to rank almost two times \XXX{overit, upresnit}). Second,
the evaluated segments can be stored together with their ranks in a database,
which can be used later to automatically evaluate unseen sentences or to tune a
system's parameters. We also discussed disadvantages of this method. The most
severe ones are that the extracted segments do not always cover the whole
sentence and that the segments are evaluated without their sentence context.

We developed an easy-to-use and modern annotation interface and conducted a
manual evaluation experiment using the proposed method. We evaluated the
systems which participated in the English-Czech direction in WMT Translation
Task. The measured the inter- and intra-annotator $\kappa$ scores (the
normalized agreement) are higher than the corresponding values in the WMT
manual evaluation, which means that our evaluation method is more robust.

To get a final score for each system's translation, we compute how often the
segments of the system were ranked better than other segments (in the context
of pairwise comparisons).  The results of evaluated systems are quite similar
to the results obtained by the official WMT judgments. However, our method is
not able to correctly distinguish some systems with very similar quality. The
Pearson correlation coefficient between the \metoda{SegRanks} scores and the
official human scores is 0.978, which is lower than correlation of some of the
best performing automatic metrics (\metric{NIST}, \metric{CDER},
\metric{ELEXR}). We manually analyzed sentences which were highly ranked in the
short segment judgments but lowly ranked in the official WMT judgment to
explain the difference. In most of these sentences, there was a badly
translated part which was, however, not covered with evaluated short segments.
The uncovered part often contained a predicate which has significant impact on
the translation quality. 

To explore the possibility of reusing the collected database to evaluate unseen
translations, we have performed several experiments. In the first one, we
evaluated unseen translations using only the ranks of the segments which were
in the database.  This, however, did not work as expected, because the obtained
scores of unseen systems were significantly overestimated. During a manual
analysis, we verified that the evaluated systems are more likely to agree on
better translations than on worse translations. Although this method cannot be
used for evaluating unseen translations, we found out that errors in machine
translation are unique.

To avoid evaluating unseen translations only on not representative subset of
short segments, we proposed another method. In this method, we evaluated unseen
translations on all the extracted segments. To approximate a rank of an unseen
segment, we took the rank of the closest segment by edit distance. This method
didn't work as well.  The approximated rank was predicted correctly using the
closest segment only in 20.6 \% cases.  In 51.2 \% of the cases, the predicted
rank was better than the original rank. The scores were overestimated again.
The important observation here is that segments are closer to better segments
than to equally good or worse segments. This is somehow consistent with the
previous finding that errors in machine translation are unique.

In another experiment, we extracted the best ranked segments from the collected
database and considered them as good translations. We used them as additional
reference translations for \metric{BLEU}. However, it did not perform better
than original \metric{BLEU} with single reference. 

In the last experiment with the collected database, we tried to use the
database to tune a machine translation system using the MERT method.  We
proposed several variants of \metoda{SegRanks} based metrics adapted for the
MERT tuning. The tuned systems were evaluated by humans against the baseline
system tuned by \metric{BLEU}. The only variant which tuned the system better
than baseline was the variant which considered unseen segments as bad and
therefore pushed the system to produce known and already evaluated segments.

\XXX{Nejake zaverecne shrnuti shrnuti}

In the second part of this thesis, we summarized the results of the WMT14
Metrics Shared Task, which assessed the quality of various automatic machine
translation metrics. Judgements collected in the WMT14 human evaluation served
as the golden truth and we checked how well the metrics predicted the
judgements at the level of individual sentences (sentence-level task) as well
as at the level of the whole test set (system-level task).

In the system-level task, we discussed differences between Spearman's rank
correlation coefficient and Pearson correlation coefficient and decided to
chose Pearson coefficient instead of Spearman's rank coefficient as being
fairer. In the segment-level task, we introduced a new notation which exactly
specifies details on Kendall's $\tau$ computation. We also discussed several
variants of Kendall's $\tau$ used in the past and proposed and used a new
variant which does not suffer shortcomings of other variants.

As in previous years, segment-level correlations are much lower than
system-level ones, reaching at most Kendall's $\tau$ of 0.45 for the best
performing metric in the best language pair. So, there is quite some research
work to be done. We are happy to see that many new metrics emerged this year,
which also underlines the importance of the Metrics Shared Task.

\XXX{Dodelat}



\section{Future Work}


\bibliographystyle{chicago}
\bibliography{references}
\listoffigures
\listoftables

\openright
\end{document}
